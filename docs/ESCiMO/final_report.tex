\documentclass[12pt,a4paper]{report}
\usepackage[utf8]{inputenc}
\usepackage{graphicx}

\renewcommand{\chaptername}{}

\title{
Final Report - ESCiMO \\
in \LaTeX{}
}

\author{
Joris Vergeer - 1585591\\
Gerben Boot - 1575754\\
\\
\\
Daniel Telgren
}

\begin{document}
\maketitle

\begin{abstract}
%%
%%TODO Rewrite parts of this
%%
Grid Manufacturing (GM) is a new production paradigm, based upon the use of standardized and modular Reconfigurable Manufacturing Systems (RMS).\cite{SICE13}
A production unit in a RMS is called an equiplet.
Each equiplet contains interchangeable modules. 
These hardware modules have software counterparts.
In this report we look at the realisation of the state machines build into the software counterparts of modules and equiplets as well as the realisation of a SCADA for the equiplet with its modules.
\end{abstract}

\tableofcontents

\chapter{Introduction}
In the 6th semester of our course Computer Science at the University of Applied Science Utrecht we participated in the Ubiquitous Computing specialisation.

\section{Ubiquitous computing}
Ubiquitous computing is a specialisation about information processing in everyday objects.\cite{wiki_ubicomp}
This years project is integrated with the "factories of the future" project.\cite{ubi_project_note}
In factories of the future the university of applied science Utrecht tries to develop an autonomous production line using Grid Manufacturing and Reconfigurable Manufacturing Systems (RMS).

\section{Equiplet, SCada, MOdules (ESCiMO)}
In our ubiquitous computing project we are developing a system to control a production unit in an RMS called equiplet. 

Equiplets contain interchangeable modules. 
You can reconfigure an equiplet by changing its modules.
Because modules can be changed on the fly, the equiplet has to dynamically adapt to its new configuration.
Due to the autonomous nature of equiplets they can receive instruction at any time.

This together creates a lot of problems we have to solve.

\chapter{Project team}
The following people have worked on ESCiMO

\begin{tabular}{l | l}
Name       & Role \\
\hline
D. Telgren & Project supervisor \\
G. Boot    & Software developer EST, MOST \\
J. Vergeer & Software developer SCADA, MOST
\end{tabular}

\chapter{Assignment}

\section{Problem description}

\subsection{Safety}
Equiplets are autonomous machines. 
They are part of a large production platform where multiple entities can request actions form the equiplet.
These requests can be received at any time.
Even when an operator is working on an equiplet.
When an operator is working on an equiplet, it has to be absolutely safe.
Modules must be powered off and have to be guarantee that they will not power up while the operator is still nearby.
To prevent this a safety mechanism has to be build.

\subsection{Synchronisation}
When instructions require actions from multiple modules both modules have to be ready.
Modules are not guaranteed to be ready at the same time. Some modules require initialisation  and/or calibration procedures while other modules are immediately ready.
To prevent instructions to be send to modules which are not ready yet a synchronisation mechanism has to be build

\subsection{SCADA}
To operate an equiplet the operator has to be able to see what is going on with the equiplet and modules. He also want an interface where he can perform minimal interaction with the equiplet. Therefore also a SCADA system has to be developed.

\section{Goals}

\section{Requirements and conditions}

\section{Final products}

\chapter{Analysis}
\section{Overview}

\section{Analysis}

\chapter{Design}

\chapter{Realization}
\section{Global phasing}

\section{Realization pre-phase/milestones}

\chapter{Final product}
\section{Result}

\section{Evaluation}

\section{Conclusion}

\section{Recommendations}

\chapter{Process and planning}
\section{Project  approach}

\section{planning}

\section{Calculation hours and costs}

\section{Project evaluation}

\chapter{Reflection}
\section{Reflection technical competences}

\section{Reflection professional competences}

\section{Profile sketch}

\chapter{Abbreviations and concepts}

\chapter{Sources}

\chapter{Attachments}

\bibliographystyle{abbrv}
\bibliography{references}
\end{document}
